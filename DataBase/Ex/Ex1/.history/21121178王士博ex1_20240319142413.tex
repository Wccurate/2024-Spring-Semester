\documentclass{ctexart}
\usepackage{geometry}
\geometry{a4paper,scale=0.8}
\usepackage{graphicx}
\usepackage{listings}
\usepackage{xcolor}
\usepackage{hyperref}

\lstset{
  basicstyle=\ttfamily,
  columns=fullflexible,
  frame=single,
  breaklines=true,
  postbreak=\mbox{\textcolor{red}{$\hookrightarrow$}\space},
}

\title{实验报告}
\author{作者姓名}
\date{\today}

\begin{document}

\maketitle

\section{实验要求}

\begin{itemize}
  \item 建立两台云服务器之间的不安防(Server 和 Client),并实现两个实例之间的SSH免密登录。
  \item 两个实例上安装MySQL,在Server上创建数据库和用户,在Client上远程连接Server和Client。
\end{itemize}

\section{实验内容}

\subsection{构建两个云服务器实例}

实验前提是已经购买了两个2GB的实例,操作系统为Ubuntu 20.04,设置两个实例的名称分别为Server和Client,共开放配置安全组,添加规则,添加MySQL(3306)。

\subsection{前置软件安装}

\begin{lstlisting}[language=bash]
sudo apt-get update
sudo apt-get install vim
sudo apt-get install openssh-server
sudo apt-get install mysql-server
\end{lstlisting}

\subsection{SSH免密登录}

在server端创建公钥和私钥。然后将id\_rsa.pub文件中的内容写入到authorized\_keys文件,就可以在server端免密码登录SSH登录。

\begin{lstlisting}[language=bash]
ssh-keygen -t rsa -C "LIXTT"
cd .ssh
cat id_rsa.pub >> authorized_keys
cat authorized_keys
\end{lstlisting}

\subsection{创建数据库和用户}

使用apt-get工具安装mysql-server之后,在server端,用vim工具更改配置文件,将mysql.conf.d文件夹中的mysqld.cnf中的bind-address从"127.0.0.1"修改为"0.0.0.0"。

\begin{lstlisting}[language=sql]
CREATE DATABASE lixintong;
CREATE USER 'client'@'172.19.0.13' IDENTIFIED BY 'database';
GRANT ALL ON *.* TO 'client'@'172.19.0.13';
FLUSH PRIVILEGES;
\end{lstlisting}

\section{收获与体会}

通过这次实验,我学会了如何在云服务器上创建实例,并实现两个实例之间的SSH免密登录。同时,我学会了如何在云服务器上创建数据库和用户,并在Client上远程连接了Server,让我收获颇丰。

\end{document}
