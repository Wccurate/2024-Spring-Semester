\documentclass{article}
\usepackage{color}
\usepackage{soul}
\usepackage{multirow}
\usepackage{pgfplots}
\usepackage{ifthen}
\usepackage[UTF8]{ctex}
\usepackage[left=2cm,right=2cm,top=2cm,bottom=2cm]{geometry}
\geometry{a4paper}
\usepackage{tikz}
\usetikzlibrary{chains}
\newcommand{\diff}{\mathop{}\!\mathrm{d}}
\usepackage{appendix} 
\usepackage{lipsum}
\usepackage{listings}
\usepackage{diagbox}
\usepackage{pdfpages}
\usepackage{xcolor}
\usepackage{pdflscape}
\usepackage{soul}
\usepackage{booktabs}
\usepackage{longtable}
\usepackage[most]{tcolorbox}
\newtcolorbox{mycolorbox}[1][]{
  sharp corners,
  colback=white, 
  colframe=black, 
  coltext=blue, 
  boxsep=5pt, 
  left=2pt, 
  right=2pt, 
  top=1pt, 
  bottom=1pt,
  breakable,
  #1 
}
\usepackage{subcaption}
\lstset{
    backgroundcolor=\color{gray!20},
    basicstyle=\ttfamily,
    commentstyle=\color{darkgray}\ttfamily,
    stringstyle=\color{red},
    showstringspaces=false,
    numbers=left,
    numberstyle=\tiny\color{gray},
    stepnumber=1,
    numbersep=10pt,
    tabsize=4,
    showspaces=false,
    showtabs=false,
    frame=single,
    captionpos=b,
    breaklines=true,
    breakatwhitespace=false,
    escapeinside={\%*}{*)},
    xleftmargin=\parindent,
    xrightmargin=\parindent,
}
\lstdefinestyle{dockerstyle}{
    language=bash,
    keywordstyle=\color{blue}\bfseries,
    morekeywords={FROM, RUN, CMD, LABEL, EXPOSE, ENV, ADD, COPY, ENTRYPOINT, VOLUME, USER, WORKDIR, ARG, ONBUILD},
}
\lstdefinestyle{pythonstyle}{
    language=Python,
    keywordstyle=\color{blue}\bfseries,
    morekeywords={import, from, as, def, return, class, self, if, elif, else, while, for, try, except, with},
}
\lstdefinestyle{cstyle}{
    language=C,
    keywordstyle=\color{blue}\bfseries,
    morekeywords={size_t, printf}, 
}
\lstdefinestyle{bashstyle}{
    language=bash,
    keywordstyle=\color{blue}\bfseries,
    morekeywords={echo,sudo ,if, then, else, fi, for, in, do, done, echo, exit, return, function},
    commentstyle=\color{green}\ttfamily,
}
\usepackage{algorithm}
\usepackage{algpseudocode}
\renewcommand{\algorithmicrequire}{\textbf{Input:}}  
\renewcommand{\algorithmicensure}{\textbf{Output:}}  
\usepackage{amsmath}
\usepackage{amsthm}
\DeclareMathOperator{\sigm}{sigm}
\usepackage{graphicx}
\usepackage{float}
\renewcommand{\vec}[1]{\boldsymbol{#1}}
\usepackage{amssymb}
\usepackage{booktabs} 
\usepackage{hyperref}
\usepackage{titlesec}
\usepackage{caption}
\usepackage{fontspec}
\usepackage{xeCJK}
\setCJKmainfont{SimSun} 
\title{\Huge 数据库(2)实验2}
\author{21121178 王士博}
\begin{document}
\begin{center}
    \Huge 数据库(2)实验2
\end{center}
\section{实验环境}
\begin{itemize}
    \item 云平台:腾讯云
    \item 云平台配置:香港地区,CPU 8核心,内存 16GB,Ubuntu 20.04。
\end{itemize}
\section{实验目的}
\section{实验步骤}
\subsection{创建一个包含docker和kubernetes的镜像}
先购买一个实例,然后通过如下的命令安装docker,这下面的指令包括了三部分:
\begin{enumerate}
    \item 
\end{enumerate}
\begin{lstlisting}[style=bashstyle]
sudo apt-get update
sudo apt-get install ca-certificates curl gnupg
sudo install -m 0755 -d /etc/apt/keyrings
sudo curl -fsSL https://download.docker.com/linux/ubuntu/gpg -o /etc/apt/keyrings/docker.asc
sudo chmod a+r /etc/apt/keyrings/docker.asc
echo \
  "deb [arch=$(dpkg --print-architecture) signed-by=/etc/apt/keyrings/docker.asc] https://download.docker.com/linux/ubuntu \
  $(. /etc/os-release && echo "$VERSION_CODENAME") stable" | \
  sudo tee /etc/apt/sources.list.d/docker.list > /dev/null
sudo apt-get update
apt list -a docker-ce
sudo apt-get install docker-ce=5:19.03.10~3-0~ubuntu-focal
sudo docker run hello-world
\end{lstlisting}
\subsection{创建kubernetes集群}
\indent 通过上面导出的镜像来创建下面三个实例,通过修
改\texttt{/etc/hostname}和\texttt{/etc/hosts}文件改变三台主机的主机名
以及配置IP地址和hostname的映射。
然后在Boss实例上开启kubernetes集群,然后在fw1和fw2上加入集群。,使用如下指令开启kubernetes集群:
\begin{lstlisting}[style=bashstyle]
sudo kubeadm init --apiserver-advertise-address=172.19.0.14 --image-repository registry.aliyuncs.com/google_containers --service-cidr=10.96.0.0/12  --pod-network-cidr=10.244.0.0/16
\end{lstlisting}
\section{编程及结果分析}
\section{实验感想}
\newpage
\appendix
\section{附录}
\end{document}