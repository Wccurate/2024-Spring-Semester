\documentclass{article}
\usepackage{color}
\usepackage{soul}
\usepackage{multirow}
\usepackage{pgfplots}
\usepackage{ifthen}
\usepackage[UTF8]{ctex}
\usepackage[left=1.0cm,right=1.0cm,top=0.5cm,bottom=0.5cm]{geometry}
\geometry{a4paper}
\usepackage{tikz}
\usetikzlibrary{chains}
\newcommand{\diff}{\mathop{}\!\mathrm{d}}
\usepackage{appendix} 
\usepackage{lipsum}
\usepackage{listings}
\usepackage{diagbox}
\usepackage{pdfpages}
\usepackage{xcolor}
\usepackage{pdflscape}
\usepackage{soul}
\usepackage{booktabs}
\usepackage{longtable}
\usepackage[most]{tcolorbox}
\newtcolorbox{mycolorbox}[1][]{
  sharp corners,
  colback=white, 
  colframe=black, 
  coltext=blue, 
  boxsep=5pt, 
  left=2pt, 
  right=2pt, 
  top=1pt, 
  bottom=1pt,
  breakable,
  #1 
}
\usepackage{subcaption}
\lstset{
    backgroundcolor=\color{gray!20},
    basicstyle=\ttfamily,
    commentstyle=\color{darkgray}\ttfamily,
    stringstyle=\color{red},
    showstringspaces=false,
    numbers=left,
    numberstyle=\tiny\color{gray},
    stepnumber=1,
    numbersep=10pt,
    tabsize=4,
    showspaces=false,
    showtabs=false,
    frame=single,
    captionpos=b,
    breaklines=true,
    breakatwhitespace=false,
    escapeinside={\%*}{*)},
    breaklines=true,
    xleftmargin=\parindent,
    xrightmargin=\parindent,
}
\lstdefinestyle{dockerstyle}{
    language=bash,
    keywordstyle=\color{blue}\bfseries,
    morekeywords={FROM, RUN, CMD, LABEL, EXPOSE, ENV, ADD, COPY, ENTRYPOINT, VOLUME, USER, WORKDIR, ARG, ONBUILD},
}
\lstdefinestyle{pythonstyle}{
    language=Python,
    keywordstyle=\color{blue}\bfseries,
    morekeywords={import, from, as, def, return, class, self, if, elif, else, while, for, try, except, with},
}
\lstdefinestyle{cstyle}{
    language=C,
    keywordstyle=\color{blue}\bfseries,
    morekeywords={size_t, printf}, 
}
\lstdefinestyle{bashstyle}{
    language=bash,
    keywordstyle=\color{blue}\bfseries,
    morekeywords={echo,sudo ,if, then, else, fi, for, in, do, done, echo, exit, return, function},
    commentstyle=\color{green}\ttfamily,
}
\usepackage{algorithm}
\usepackage{algpseudocode}
\renewcommand{\algorithmicrequire}{\textbf{Input:}}  
\renewcommand{\algorithmicensure}{\textbf{Output:}}  
\usepackage{amsmath}
\usepackage{amsthm}
\DeclareMathOperator{\sigm}{sigm}
\usepackage{graphicx}
\usepackage{float}
\renewcommand{\vec}[1]{\boldsymbol{#1}}
\usepackage{amssymb}
\usepackage{booktabs} 
\usepackage{hyperref}
\usepackage{titlesec}
\usepackage{caption}
\usepackage{fontspec}
\usepackage{xeCJK}
\setCJKmainfont{SimSun} 
\title{\Huge 数据库(2)实验2}
\author{21121178 王士博}
\begin{document}
\begin{center}
    \textbf{\huge 数据库(2)实验3:HDFS分布式文件系统和SparkSQL实验}
\end{center}
\begin{center}
    \textbf{\large \textbf{学号:21121178 \quad 姓名:王士博 \quad 指导教师:刘洋}}
\end{center}
\hrulefill
\section{实验环境}
\begin{itemize}
    \item 腾讯云服务器配置:竞价实例,南京地区,标准型SA5,4核 8GB,Ubuntu 20.04。
\end{itemize}
\section{实验目的}
\begin{enumerate}
    \item 分布式文件系统HDFS。
    \item 学会使用SparkSQL。
\end{enumerate}
\section{实验步骤}
\subsection{生成安装所有依赖的镜像}
\begin{enumerate}
    \item 安装JDK8和JRE(其实直接安装JDK会包括JRE),使用\texttt{apt-get}安装之后需要在\texttt{.bashrc}
    中显示给出\texttt{JAVA\_HOME}的路径,并且使用\texttt{source}应用更改。
    \begin{figure}[H]
        \centering
        \includegraphics[width=0.7\textwidth]{java.png}
    \end{figure}
    \item 使用scp上传本地的Hadoop和Spark压缩包到服务器并且解压到软件包目录\texttt{/usr/local/}下
    ,更改文件的owner为当前的用户,并且赋予所有的权限,然后cd进可执行文件使用\texttt{-version}来测试
    Hadoop安装包的完整性以及安装的成功性,并且使用\texttt{SparkPi}来测试Spark的安装成功性(后面)。
    \begin{figure}[H]
        \begin{subfigure}{0.6\textwidth}
            \centering
            \includegraphics[width=\textwidth]{hadoopinstall.png}
            \caption*{hadoop version}
        \end{subfigure}
        \hfill
        \begin{subfigure}{0.35\textwidth}
            \centering
            \includegraphics[width=\textwidth]{sparkpi.png}
            \caption*{SparkPi}
        \end{subfigure}
    \end{figure}
    \item 需要配置\texttt{core-site.xml}和\texttt{hdfs-site.xml},\texttt{yarn-site.xml},\texttt{mapred-site.xml},
    以及\texttt{slaves}文件,现将工人设置成自己的hostname以及剩下两个机器的hostname,然后删除掉
    \texttt{known\_hosts}文件,然后将这个实例创建成镜像。
    \begin{figure}[H]
        \centering
        \includegraphics[width=0.7\textwidth]{mkimage.png}
    \end{figure}
    
\end{enumerate}
\subsection{创建三个实例启动分布式文件系统HDFS}
\begin{enumerate}
    \item 设置三个机器\texttt{master},\texttt{slave01}和\texttt{slave02}的\texttt{hostname}和\texttt{hosts}
    文件,使得三个主机可以使用hostname相互无密码登陆。
    \item 在\texttt{master}上启动HDFS,首先格式化HDFS,然后启动HDFS,使用\texttt{jps}查看
    运行中的JVM实例也就是运行的进程。
    \begin{figure}
        \begin{subfigure}{0.6\textwidth}
            \centering
            \includegraphics[width=\textwidth]{starthadoop.png}
            \caption*{启动HDFS}
        \end{subfigure}
        \hfill
        \begin{subfigure}{0.35\textwidth}
            \centering
            \includegraphics[width=\textwidth]{jps.png}
            \caption*{jps查看运行的JVM实例}
        \end{subfigure}
    \end{figure}
    \item 上传文件到HDFS,首先创建一个文件夹,然后上传一个文件到HDFS,然后在三个实例上查看查看HDFS上的文件是否同步。
    \begin{figure}[H]
        \centering
        \includegraphics[width=0.65\textwidth]{uploadfile.png}
    \end{figure}
    \item 使用Spark创建一个数据库\texttt{sparktest},并且使用\texttt{people.json}文件创建一个表
    \texttt{tablejson},然后使用\texttt{select * from tablejson}来查看是否创建成功。
    \begin{figure}[H]
        \begin{subfigure}{0.5\textwidth}
            \centering
            \includegraphics[width=\textwidth]{databases.png}
            \caption*{Database}
        \end{subfigure}
        \hfill
        \begin{subfigure}{0.45\textwidth}
            \centering
            \includegraphics[width=\textwidth]{table.png}
            \caption*{Tablejson}
        \end{subfigure}
    \end{figure}
\end{enumerate}
\subsection{Titanic数据集的分析}
\begin{enumerate}
    \item 将csv转换成json格式,并且使用scp上传到服务器,使用下面的python代码实现。
    \begin{lstlisting}[style=pythonstyle]
data = pd.read_csv('titanic.csv')
json_path = 'titanic.json'
data.to_json(json_path, orient='records', lines=True)
    \end{lstlisting}
    \item 将\texttt{titanic.json}上传到HDFS中。
    \begin{figure}
        \centering
        \includegraphics[width=0.7\textwidth]{uploadtitanic.png}
    \end{figure}
    \item 使用SparkSQL读取\texttt{titanic.json}文件,然后创建一个表\texttt{titanic}
    然后进行数据分析:
    \begin{enumerate}
        \item 
    \end{enumerate}
    
\end{enumerate}
\section{实验结论}

\end{document}