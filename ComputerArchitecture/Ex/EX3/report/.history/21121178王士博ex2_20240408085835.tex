\documentclass{article}
\usepackage{color}
\usepackage{soul}
\usepackage{multirow}
\usepackage{pgfplots}
\usepackage{ifthen}
\usepackage[UTF8]{ctex}
\usepackage[left=2cm,right=2cm,top=2cm,bottom=2cm]{geometry}
\geometry{a4paper}
\usepackage{tikz}
\usetikzlibrary{chains}
\newcommand{\diff}{\mathop{}\!\mathrm{d}}
\usepackage{appendix} 
\usepackage{lipsum}
\usepackage{listings}
\usepackage{diagbox}
\usepackage{pdfpages}
\usepackage{xcolor}
\usepackage{pdflscape}
\usepackage{soul}
\usepackage{booktabs}
\usepackage{longtable}
\usepackage[most]{tcolorbox}
\newtcolorbox{mycolorbox}[1][]{
  sharp corners,
  colback=white, 
  colframe=black, 
  coltext=blue, 
  boxsep=5pt, 
  left=2pt, 
  right=2pt, 
  top=1pt, 
  bottom=1pt,
  breakable,
  #1 
}
\usepackage{subcaption}
\lstset{
    backgroundcolor=\color{gray!20},
    basicstyle=\ttfamily,
    commentstyle=\color{darkgray}\ttfamily,
    stringstyle=\color{red},
    showstringspaces=false,
    numbers=left,
    numberstyle=\tiny\color{gray},
    stepnumber=1,
    numbersep=10pt,
    tabsize=4,
    showspaces=false,
    showtabs=false,
    frame=single,
    captionpos=b,
    breaklines=true,
    breakatwhitespace=false,
    escapeinside={\%*}{*)},
    xleftmargin=\parindent,
    xrightmargin=\parindent,
}
\lstdefinestyle{dockerstyle}{
    language=bash,
    keywordstyle=\color{blue}\bfseries,
    morekeywords={FROM, RUN, CMD, LABEL, EXPOSE, ENV, ADD, COPY, ENTRYPOINT, VOLUME, USER, WORKDIR, ARG, ONBUILD},
}
\lstdefinestyle{pythonstyle}{
    language=Python,
    keywordstyle=\color{blue}\bfseries,
    morekeywords={import, from, as, def, return, class, self, if, elif, else, while, for, try, except, with},
}
\lstdefinestyle{cstyle}{
    language=C,
    keywordstyle=\color{blue}\bfseries,
    morekeywords={size_t, printf},
}
\lstdefinestyle{bashstyle}{
    language=bash,
    keywordstyle=\color{blue}\bfseries,
    morekeywords={if, then, else, fi, for, in, do, done, echo, exit, return, function},
    commentstyle=\color{green}\ttfamily,
}
\usepackage{algorithm}
\usepackage{algpseudocode}
\renewcommand{\algorithmicrequire}{\textbf{Input:}}  
\renewcommand{\algorithmicensure}{\textbf{Output:}}  
\usepackage{amsmath}
\usepackage{amsthm}
\DeclareMathOperator{\sigm}{sigm}
\usepackage{graphicx}
\usepackage{float}
\renewcommand{\vec}[1]{\boldsymbol{#1}}
\usepackage{amssymb}
\usepackage{booktabs} 
\usepackage{hyperref}
\usepackage{titlesec}
\usepackage{caption}
\usepackage{fontspec}
\usepackage{xeCJK}
\setCJKmainfont{SimSun} 
\title{\Huge CUDA\&CUDA和OpenMP混合编程实验报告}
\author{21121178 王士博}
\begin{document}
\maketitle
\section{实验环境}
\textbf{Windows11 Professional 22H2 x64:} \\
\\
\indent \indent \indent \begin{minipage}[H]{0.7\textwidth}
    \begin{itemize}
        \item \textbf{处理器:}Intel(R) Core(TM) i7-11800H CPU @ 4.60GHz 16线程
        \item \textbf{内存:}32.0 GB
        \item \textbf{显卡:}NVIDIA GeForce RTX 3070
        \item \textbf{显卡驱动版本:}11.6
        \item \textbf{编程语言:}C++
        \item \textbf{编译器:}nvcc
        \item \textbf{编程平台:}Visual Studio 2019
    \end{itemize}
\end{minipage}
\section{实验目的}
\begin{enumerate}
    \item 了解CUDA的安装方式和注意问题。通过简洁明了地列出CUDA安装的步骤和关键注意事项,目的在于为CUDA新用户提供一个清晰、易于遵循的安装指南。这不仅帮助用户高效完成CUDA环境的配置,也旨在通过分享实践经验,降低并行编程技术的入门门槛,尤其是对于那些首次尝试在个人计算机上进行CUDA编程的用户。
    \item 使用CUDA或者CUDA和OpenMP混合编程实现矩阵乘法并且根据自己的电脑配置进行优化。现矩阵乘法的CUDA版本以及可能的OpenMP-CUDA混合编程版本,并基于个人机器的具体配置,从多个角度对比其加速比。此实验目的在于探究CUDA和OpenMP-CUDA混合编程技术在执行复杂运算任务时的性能差异,以及评估在不同计算资源配置下的执行效率。通过实验,旨在深入理解GPU并行计算的优势与局限,以及混合编程模式在提升计算性能方面的潜力,同时,通过附上源码,增加实验的可重复性和透明度,为并行计算领域的学习者和研究者提供参考。
\end{enumerate}
\section{实验步骤}
\subsection{CUDA的安装}
\begin{enumerate}
    
    \item \textbf{下载CUDA Toolkit:}首先,访问NVIDIA官网,下载适用于自己操作系统的CUDA Toolkit。CUDA Toolkit是一个用于开发CUDA程序的软件包,包含了CUDA编译器、CUDA库、CUDA工具等。下载地址:\url{https://developer.nvidia.com/cuda-downloads}。
\end{enumerate}
\section{实验感想}

\newpage
\appendix
\section{源代码}
\end{document}