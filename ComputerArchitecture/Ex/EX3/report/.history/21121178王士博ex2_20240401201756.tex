\documentclass{article}
\usepackage{color}
\usepackage{soul}
\usepackage{multirow}
\usepackage{pgfplots}
\usepackage{ifthen}
\usepackage[UTF8]{ctex}
\usepackage[left=2cm,right=2cm,top=2cm,bottom=2cm]{geometry}
\geometry{a4paper}
\usepackage{tikz}
\usetikzlibrary{chains}
\newcommand{\diff}{\mathop{}\!\mathrm{d}}
\usepackage{appendix} 
\usepackage{lipsum}
\usepackage{listings}
\usepackage{diagbox}
\usepackage{pdfpages}
\usepackage{xcolor}
\usepackage{pdflscape}
\usepackage{soul}
\usepackage{booktabs}
\usepackage{longtable}
\usepackage[most]{tcolorbox}
% \tcbuselibrary{breakable}
\newtcolorbox{mycolorbox}[1][]{
  sharp corners,
  colback=white, 
  colframe=black, 
  coltext=blue, 
  boxsep=5pt, 
  left=2pt, 
  right=2pt, 
  top=1pt, 
  bottom=1pt,
  breakable,
  #1 
}
\usepackage{subcaption}

% 通用设置
\lstset{
    %backgroundcolor=\color{white},
    backgroundcolor=\color{gray!20},
    basicstyle=\ttfamily,
    commentstyle=\color{darkgray}\ttfamily,
    stringstyle=\color{red},
    showstringspaces=false,
    numbers=left,
    numberstyle=\tiny\color{gray},
    stepnumber=1,
    numbersep=10pt,
    tabsize=4,
    showspaces=false,
    showtabs=false,
    frame=single,
    captionpos=b,
    breaklines=true,
    breakatwhitespace=false,
    escapeinside={\%*}{*)},
    xleftmargin=\parindent,
    xrightmargin=\parindent,
}

% Dockerfile 样式
\lstdefinestyle{dockerstyle}{
    language=bash,
    keywordstyle=\color{blue}\bfseries,
    morekeywords={FROM, RUN, CMD, LABEL, EXPOSE, ENV, ADD, COPY, ENTRYPOINT, VOLUME, USER, WORKDIR, ARG, ONBUILD},
}

% Python 样式
\lstdefinestyle{pythonstyle}{
    language=Python,
    keywordstyle=\color{blue}\bfseries,
    morekeywords={import, from, as, def, return, class, self, if, elif, else, while, for, try, except, with},
}
\lstdefinestyle{cstyle}{
    language=C,
    keywordstyle=\color{blue}\bfseries,
    morekeywords={size_t, printf}, % 只有在需要额外关键字时才取消注释并添加
}
% Bash 样式
\lstdefinestyle{bashstyle}{
    language=bash,
    keywordstyle=\color{blue}\bfseries,
    morekeywords={if, then, else, fi, for, in, do, done, echo, exit, return, function},
    commentstyle=\color{green}\ttfamily,
}
\usepackage{algorithm}
\usepackage{algpseudocode}
\renewcommand{\algorithmicrequire}{\textbf{Input:}}  
\renewcommand{\algorithmicensure}{\textbf{Output:}}  
\usepackage{amsmath}
\usepackage{amsthm}
\DeclareMathOperator{\sigm}{sigm}
\usepackage{graphicx}
\usepackage{float}
\renewcommand{\vec}[1]{\boldsymbol{#1}}
\usepackage{amssymb}
\usepackage{booktabs} 
\usepackage{hyperref}
\usepackage{titlesec}
\usepackage{caption}
\usepackage{fontspec}
\usepackage{xeCJK}
\setCJKmainfont{SimSun} 
\title{\Huge 内存对性能影响实验文档}
\author{王士博 \\
\texttt{\href{mailto:wangshibo@shu.edu.cn}{wangshibo@shu.edu.cn}}
}
\begin{document}
\maketitle
\section{实验环境和软件}
\begin{enumerate}
  \item 宿主机配置:Windows11,Intel(R) Core(TM) i7-11800H CPU,32.0 GB RAM,NVIDIA GeForce RTX 3070 GPU;
  \item 实验环境:docker镜像服从下面的dockerfile配置;
   \begin{lstlisting}[style=dockerstyle,label=lst:dockerfile]
FROM ubuntu:22.04

ENV DEBIAN_FRONTEND=noninteractive
RUN apt -y update && apt -y upgrade && \
    apt -y install build-essential git m4 scons zlib1g zlib1g-dev \
    libprotobuf-dev protobuf-compiler libprotoc-dev libgoogle-perftools-dev \
    python3-dev doxygen libboost-all-dev libhdf5-serial-dev python3-pydot \
    libpng-dev libelf-dev pkg-config pip python3-venv black

RUN pip install mypy pre-commit
   \end{lstlisting}
  \item 仿真软件:Gem5\cite{Binkert2011gem5};
\end{enumerate}
\section{实验内容}
\subsection{Gem5简介和安装}
Gem5对环境的要求较为严苛,在GitHub仓库的Issue中很大一部分是对编译失败的问题的讨论。所以这里展示
使用docker镜像来安装Gem5的过程。由于仓库中的Ubuntu20.04镜像链接失效,这里使用Ubuntu22.04镜像
安装。镜像地址为\url{ghcr.io/gem5/ubuntu-22.04_all-dependencies:v23-0},dockerfile如实验环境
\ref{lst:dockerfile}所示,使用如下的命令来下拉和创建实例,并且在宿主机使用git下拉Gem5的源码。
\begin{lstlisting}[style=bashstyle]
docker pull ghcr.io/gem5/ubuntu-22.04_all-dependencies:v23-0
#docker build -t your_image_name:tag . for dockerfile
git clone https://github.com/gem5/gem5
docker run --volume gem5:/gem5 -it ghcr.io/gem5/ubuntu-22.04_all-dependencies:v23-0
\end{lstlisting}
进入虚拟终端之后进入gem5目录之后,执行如下的命令使用scons构建gem5,将线程数设置为宿
主机的线程数,在后	续的构建中可能CPU会长期占用100\%,但是如果线程数过少构建缓
慢,实测16线程需要30min左右。
\begin{lstlisting}[style=bashstyle]
cd /your/path/to/gem5
python3 `which scons` /build/X86/gem5.opt -j$(nproc)
>>scons:successfully build target
\end{lstlisting}
\subsection{构建模拟器}
构建一个Gem5的模拟系统,需要下面几个部件。第一个是CPU及其相关的配置如下:
\begin{lstlisting}[style=pythonstyle]
system = System()
system.clk_domain = SrcClockDomain()
if not options or not options.clk:
    system.clk_domain.clock = '1GHz'
else:
    system.clk_domain.clock = options.clk
system.clk_domain.voltage_domain = VoltageDomain()
\end{lstlisting}

第二个是内存的配置,这里使用了DDR3\_1600\_8x8的配置,如下所示:
\begin{lstlisting}[style=pythonstyle]
system.mem_mode = 'timing'
if not options or not options.mem_size:
    system.mem_ranges = [AddrRange('512MB')]
else:
    system.mem_ranges = [AddrRange(options.mem_size)]
\end{lstlisting}

第三个需要配置的是Cache,这里使用了L1和L2的Cache,这里的Cache需要继承自Cache基
类自己进行细节的配置,如下所示:
\begin{lstlisting}[style=pythonstyle]
system.cpu = TimingSimpleCPU()
system.cpu.icache = L1ICache(options)
system.cpu.dcache = L1DCache(options)
system.cpu.icache.connectCPU(system.cpu)
system.cpu.dcache.connectCPU(system.cpu)
system.l2bus = L2XBar()
system.cpu.icache.connectBus(system.l2bus)
system.cpu.dcache.connectBus(system.l2bus)
system.l2cache = L2Cache(options)
system.l2cache.connectCPUSideBus(system.l2bus)
system.membus = SystemXBar()
system.l2cache.connectMemSideBus(system.membus)
if not options or not options.cache_block:
    system.cache_line_size = 64
else:
    system.cache_line_size = options.cache_block
\end{lstlisting}

第四步需要配置的是内存控制器,这里使用了DDR3\_1600\_8x8的配置,如下所示:
\begin{lstlisting}[style=pythonstyle]
system.mem_ctrl = MemCtrl()
system.mem_ctrl.dram = DDR3_1600_8x8()
system.mem_ctrl.dram.range = system.mem_ranges[0]
system.mem_ctrl.port = system.membus.mem_side_ports
\end{lstlisting}

最后可以指定一个测试程序进行测试,这个程序一定是静态编译之后的结果,比如test.o文件。下面是这个程序的
一个使用实例。其中需要保证测试文件要在gem5/tests/test\_progs的目录之下以保证可以正常识别,
并且上面的配置文件需要放在gem5/configs目录之下。在执行测试时使用构建出的/build/X86/gem5.opt二进制
可执行文件执行传入参数的配置文件进行测试。下面是一个测试的完整流程。
\begin{lstlisting}[style=bashstyle]
vim tests/test-progs/bandwidth_test.c
gcc -static tests/test-progs/bandwidth_test.c -o tests/test-progs/bandwidth_test -lm
/build/X86/gem5.opt configs/exps/cached_cpu.py --clk='1GHz' --l1i_size='16kB' --l1d_size='128kB' --cache_block=16 --l2_size='512kB' --test_dir='tests/test-progs/bandwidth_test' --mem_size='512MB'
\end{lstlisting}

通过执行\texttt{/build/X86/gem5.opt configs/exps/cached\_cpu.py --help}来查看参数情况。
\begin{lstlisting}[style=bashstyle]
>>Usage: cached_cpu.py [options]
  Options:
    -h, --help            show this help message and exit
    --clk=CLK             CPU clk. Default: 1GHz
    --cache_block=CACHE_BLOCK
                          CPU cache block. Default: 64
    --l1i_size=L1I_SIZE   L1 instruction cache size. Default: 16kB.
    --l1d_size=L1D_SIZE   L1 data cache size. Default: Default: 64kB.
    --l2_size=L2_SIZE     L2 cache size. Default: 256kB.
    --mem_size=MEM_SIZE   Memory size. Default: 512MB.
    --test_dir=TEST_DIR   Test directory. Default: tests/test-
                          progs/hello/bin/x86/linux/hello
    --ddr4=DDR4           DDR4 with 2400Mhz. Default:DDR3 1600Mhz
\end{lstlisting}
\subsection{实验程序}
\begin{enumerate}
  \item 矩阵乘法运算来测试缓存和缓存行对性能的影响:
  \begin{lstlisting}[style=cstyle]
    #include <stdio.h>
    #include <stdlib.h>
    #include <time.h>
    #define N 128
    void matrix_multiply(double **a, double **b, double **result) {
        for (int i = 0; i < N; i++) {
            for (int j = 0; j < N; j++) {
                result[i][j] = 0.0;
                for (int k = 0; k < N; k++) {
                    result[i][j] += a[i][k] * b[k][j];
                }
            }
        }
    }
    int main() {
        double **a = (double **)malloc(N * sizeof(double *));
        double **b = (double **)malloc(N * sizeof(double *));
        double **result = (double **)malloc(N * sizeof(double *));
        for (int i = 0; i < N; i++) {
            a[i] = (double *)malloc(N * sizeof(double));
            b[i] = (double *)malloc(N * sizeof(double));
            result[i] = (double *)malloc(N * sizeof(double));
        }
        srand(time(NULL));
        for (int i = 0; i < N; i++) {
            for (int j = 0; j < N; j++) {
                a[i][j] = rand() / (double)RAND_MAX;
                b[i][j] = rand() / (double)RAND_MAX;
            }
        }
        matrix_multiply(a, b, result);
        printf("Matrix multiplication completed.\n");
        for (int i = 0; i < N; i++) {
            free(a[i]);
            free(b[i]);
            free(result[i]);
        }
        free(a);
        free(b);
        free(result);
        return 0;
    }
  \end{lstlisting}
  \item 使用内存开辟和数组遍历来测试内存大小对性能的影响,这个程序会申请比主存更多的内存空间,
  这会迫使进行主存和外存的交换,将主存的内容移动到外存上,并且之后进行遍历,迫使程序从外存上读取数据:
  \begin{lstlisting}[style=cstyle]
    #include <stdio.h>
    #include <stdlib.h>
    #define SIZE 80000000
    int main() {
        long long *largeArray = malloc(SIZE * sizeof(long long));
        if (largeArray == NULL) {
            printf("Failed to allocate memory.\n");
            return 1;
        }
        for (long long i = 0; i < SIZE; i++) {
            largeArray[i] = i;
        }
        printf("Large dataset manipulation completed.\n");
        free(largeArray);
        return 0;
    }
  \end{lstlisting}
  \item 使用加法程序测试内存频率/带宽对性能的影响,因为CPU的加法运算非常快,而且CPU的时钟频率远高于内存的频率
  所以在CPU频率较高并且没有Cache的情况下,这个程序的瓶颈在于主存的频率(主存读取数据的快慢):
  \begin{lstlisting}[style=cstyle]
    #include <stdio.h>
    #include <stdlib.h>
    #include <time.h>
    #define ARRAY_SIZE (1024 * 1024 * 64) 
    double calculateSum(double *array, int size) {
        double sum = 0.0;
        for (int i = 0; i < size; ++i) {
            sum += array[i];
        }
        return sum;
    }
    int main() {
        double *array = (double *)malloc(ARRAY_SIZE * sizeof(double));
        if (array == NULL) {
            fprintf(stderr, "Memory allocation failed.\n");
            return 1;
        }
        srand(time(NULL));
        for (int i = 0; i < ARRAY_SIZE; ++i) {
            array[i] = (double)rand() / RAND_MAX * 100;
        }
        clock_t start = clock();
        double sum = calculateSum(array, ARRAY_SIZE);
        clock_t end = clock();
        printf("Array sum: %f\n", sum);
        printf("Time taken: %f seconds\n", (double)(end - start) / CLOCKS_PER_SEC);
        free(array);
        return 0;
    }
  \end{lstlisting}
\end{enumerate}
\section{实验结果}
\begin{enumerate}
  \item 矩阵乘法运算实验结果如下:
  \begin{table}[h!]
    \centering
    \small
    \resizebox{\textwidth}{!}{
    \begin{tabular}[width=1.2\textwidth]{cccccccccc}
    \toprule
    配置编号 & 主存大小 & L1 Data Cache & L1 Instruction Cache & L2 Cache & Cache Blocks & 主频率 (MHz) & 运行时间 (s) & Cache命中次数 & CPI (周期/指令) \\
    \midrule
    1 & 512MB & 128KB & 32KB & 512KB & 8 & 1600 & 0.43 & 47123934 & 4.09 \\
    2 & 512MB & 128KB & 32KB & 512KB & 16 & 1600 & 0.40 & 48365326 & 3.85 \\
    3 & 512MB & 256KB & 32KB & 512KB & 8 & 1600 & 0.43 & 47192301 & 4.10 \\
    4 & 512MB & 128KB & 32KB & 512KB & 8 & 2400 & 0.43 & 47023931 & 4.09 \\
    5 & 512MB & 128KB & 32KB & 1024KB & 8 & 1600 & 0.43 & 47343133 & 4.09 \\
    6 & 512MB & 256KB & 32KB & 512KB & 8 & 1600 & 0.43 & 47239412 & 4.09 \\
    7 & 512MB & 128KB & 16KB & 512KB & 8 & 1600 & 0.43 & 47297938 & 4.09 \\
    8 & 512MB & 128KB & 32KB & 512KB & 32 & 1600 & 0.38 & 28817136 & 3.71 \\
    9 & 512MB & 16KB & 32KB & 16KB & 8 & 1600 & 0.60 & 47173460 & 5.71 \\
    10(3Ghz) & 512MB & 16KB & 32KB & 16KB & 8 & 1600 & 0.30 & 47173460 & 10.71 \\
    \bottomrule
    \end{tabular}}
    \caption{Cache测试}
    \label{table:cache}
    \end{table}

    对于矩阵乘法程序而言,以下配置参数对性能影响显著:

\begin{enumerate}
    \item \textbf{L1 数据缓存(L1 Data Cache)}:程序对L1数据缓存大小非常敏感。配置9表明了L1数据缓存小(16KB)时,执行时间长且CPI高,性能下降显著。
    \item \textbf{缓存块数(Cache Blocks)}:较高的缓存块数可以改善性能。配置2与配置8虽然L1数据缓存大小相同,但更多的缓存块数(分别为16和32)带来了更低的CPI。
    \item \textbf{L2 缓存(L2 Cache)}:L2缓存大小对性能有正面影响。配置5相对配置1的L2缓存加倍,使得性能得到了显著提升。
    \item \textbf{主频(MHz)}:仅增加主频并不足以显著提高性能。配置4的高主频(2400MHz)并未带来预期的性能提升,这表明缓存效率更为关键。
\end{enumerate}

综上所述,矩阵乘法程序对缓存参数非常敏感,适当增加缓存大小和块数可以显著提高程序执行效率。
  \item 内存大小对性能的影响如下:
  \begin{table}[h!]
    \centering
    \resizebox{0.5\textwidth}{!}{
    \begin{tabular}[width=1.2\textwidth]{cccc}
    \toprule
    配置编号 & 主存大小(MB) & 数组长度($10^{7}$) & 运行时间 (s) \\
    \midrule
    1 & 512MB & 8 & 6.00  \\
    2 & 1024MB & 8 & 4.12  \\
    3 & 512MB & 4 & 2.01  \\
    4 & 512MB & 4 & 2.04  \\
    5 & 512MB & 12 & 15.12  \\
    \bottomrule
    \end{tabular}}
    \caption{内存大小测试}
    \label{table:mem_size}
    \end{table}

    表\ref{table:mem_size}展示了不同主存大小对强制内存外存交换程序性能的影响。以下为分析结果:

\begin{enumerate}
    \item \textbf{主存大小(Memory Size)}:主存大小的增加能显著减少执行时间,如配置2相比配置1,在故障次数相同的情况下,主存增加一倍,执行时间从6.00秒减少到4.12秒。
    \item \textbf{故障次数(Faults)}:故障次数的减少直接影响执行时间的缩短。配置3和配置4在主存大小相同的情况下,故障次数减少,执行时间由6.00秒显著减少到约2.02秒。
    \item \textbf{执行时间(Execution Time)}:当故障次数显著增加时,执行时间会大幅上升,如配置5的故障次数是配置1的1.5倍,导致执行时间从6.00秒增加到15.12秒。
\end{enumerate}
  \item 内存频率对性能的影响如下,对于此程序采用无缓存的单一CPU和主存的结构进行测试:
  \begin{table}[h!]
    \centering
    \small
    \resizebox{0.5\textwidth}{!}{
    \begin{tabular}[width=1.2\textwidth]{cccc}
    \toprule
    配置编号 & CPU频率(Ghz) & 主存频率(Mhz) & 运行时间(s) \\
    \midrule
    1 & 1 & DDR3/1600 & 1.02  \\
    2 & 1 & DDR4/2400 & 0.71  \\
    3 & 3 & DDR3/1600 & 1.01  \\
    4 & 3 & DDR4/2400 & 0.61  \\
    \bottomrule
    \end{tabular}}
    \caption{内存频率测试}
    \label{table:mem_fre}
    \end{table}

    在大数组求和任务中,CPU与主存频率对性能有显著影响。表3提供了不同配置下的执行时间。以下为详细分析:

\begin{enumerate}
    \item \textbf{CPU频率(GHz)}:对于CPU密集型任务,理论上CPU频率越高,执行速度越快。然而,配置1和配置3的比较显示,即使CPU频率提高了两倍,执行时间的改进并不明显(1.02秒对比1.01秒)。
    \item \textbf{主存频率(MHz)}:配置1和配置2的比较表明,提高主存频率可以显著减少执行时间(从1.02秒减少到0.71秒),突显了数据传输速率在大数组求和程序中的重要性。
    \item \textbf{执行时间(s)}:配置4提供了最短的执行时间(0.61秒),显示了同时提高CPU和主存频率可以协同工作,以达到最佳的程序性能。
\end{enumerate}
\end{enumerate}
\section{实验结论}
\begin{enumerate}
  \item \textbf{主存大小}
  \begin{itemize}
      \item \textbf{敏感应用及分析:}大型数据库和虚拟机环境特别敏感于主存大小。大型数据库需要将大量数据加载到内存中以提高查询效率,而虚拟机环境中,每个虚拟机分配的内存越多,能够并行运行的虚拟机数量就越多,系统性能也越好。
      \item \textbf{影响:}不足的主存大小会导致系统频繁地使用磁盘空间作为虚拟内存,极大地增加了数据访问的延迟,降低了系统的整体性能。
      \item \textbf{瓶颈原因:}主存不足时,系统无法将所有需要的数据和程序同时加载到内存中,必须从较慢的磁盘存储中调入数据,这增加了处理时间。
      \item \textbf{时间浪费:}在频繁的数据交换过程中,大量时间花费在等待磁盘I/O操作上,CPU在这段时间内可能处于闲置状态,导致资源浪费。
  \end{itemize}
  
  \item \textbf{主存频率(主存带宽)}
  \begin{itemize}
      \item \textbf{敏感应用及分析:}视频编辑、3D渲染和科学计算应用对主存频率特别敏感。这些应用需要快速处理和传输大量数据,高带宽可以显著提升这些应用的执行效率。
      \item \textbf{影响:}如果主存频率过低,即使CPU有高速的处理能力,数据传输的瓶颈也会限制整体性能,导致处理速度下降。
      \item \textbf{瓶颈原因:}主存频率低意味着数据在内存和CPU之间的传输速度慢,处理器需要等待数据,无法充分利用其计算能力。
      \item \textbf{时间浪费:}处理器在等待内存数据时的闲置时间增加,导致处理效率降低。
  \end{itemize}
  
  \item \textbf{缓存大小}
  \begin{itemize}
      \item \textbf{敏感应用及分析:}高性能计算(HPC)和大型在线事务处理(OLTP)系统对缓存大小极为敏感。这些系统需要频繁访问大量数据,较大的缓存可以减少对慢速主存的访问次数,提高处理速度。
      \item \textbf{影响:}缓存大小不足会导致高缓存失效率,增加对主存的访问需求,从而增加数据访问延迟,降低系统性能。
      \item \textbf{瓶颈原因:}当缓存无法容纳足够的数据以供CPU快速访问时,必须从相对较慢的主存中加载数据,导致处理速度受限。
      \item \textbf{时间浪费:}CPU在等待必要数据从主存传输到缓存时的闲置时间增加,效率降低。
  \end{itemize}
  
  \item \textbf{缓存级数}
  \begin{itemize}
      \item \textbf{敏感应用及分析:}复杂的软件应用,如数据库管理系统和大规模集成电路(VLSI)设计工具,对缓存的级数特别敏感。多级缓存可以有效地减少对主存的访问次数,提高数据访问速度。
      \item \textbf{影响:}缓存级数较少可能导致数据访问效率降低,因为它减少了存储频繁访问数据的机会,增加了对慢速主存的依赖。
      \item \textbf{瓶颈原因:}缺乏足够级数的缓存意味着数据在缓存和主存之间的移动次数增加,导致数据访问延迟。
      \item \textbf{时间浪费:}数据在缓存级别之间移动所需的时间增加,减慢了处理速度,CPU等待数据的时间增长。
  \end{itemize}
  
  \item \textbf{缓存行大小}
  \begin{itemize}
      \item \textbf{敏感应用及分析:}内存密集型应用,如流式处理和大数据分析,对缓存行大小特别敏感。合适的缓存行大小可以减少缓存失效次数和不必要的数据加载。
      \item \textbf{影响:}不适当的缓存行大小会导致缓存利用率低下,增加缓存失效率,从而需要更频繁地从主存中加载数据。
      \item \textbf{瓶颈原因:}缓存行太大可能导致加载过多不必要的数据,浪费缓存空间;太小则增加了缓存管理的开销。
      \item \textbf{时间浪费:}不合理的缓存行大小导致的高缓存失效率增加了CPU等待数据的时间,降低了处理效率。
  \end{itemize}
\end{enumerate}
\end{document}